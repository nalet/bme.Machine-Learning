% --------------------------------------------------------------
% This is all preamble stuff that you don't have to worry about.
% Head down to where it says "Start here"
% --------------------------------------------------------------

\documentclass[12pt]{article}

\usepackage[margin=1in]{geometry}
\usepackage{amsmath,amsthm,amssymb}
\usepackage{graphicx} %This allows to include eps figures
\usepackage{subcaption}
\usepackage[section]{placeins}
\usepackage{layout}
\usepackage{etoolbox}
\usepackage{mathabx}
% This is to include code
\usepackage{listings}
\usepackage{xcolor}
\definecolor{dkgreen}{rgb}{0,0.6,0}
\definecolor{gray}{rgb}{0.5,0.5,0.5}
\definecolor{mauve}{rgb}{0.58,0,0.82}
\lstdefinestyle{Python}{
    language        = Python,
    basicstyle      = \ttfamily,
    keywordstyle    = \color{blue},
    keywordstyle    = [2] \color{teal}, % just to check that it works
    stringstyle     = \color{green},
    commentstyle    = \color{red}\ttfamily
}

\newcommand{\N}{\mathbb{N}}
\newcommand{\Z}{\mathbb{Z}}

\newenvironment{theorem}[2][Theorem]{\begin{trivlist}
\item[\hskip \labelsep {\bfseries #1}\hskip \labelsep {\bfseries #2.}]}{\end{trivlist}}
\newenvironment{lemma}[2][Lemma]{\begin{trivlist}
\item[\hskip \labelsep {\bfseries #1}\hskip \labelsep {\bfseries #2.}]}{\end{trivlist}}
\newenvironment{exercise}[2][Exercise]{\begin{trivlist}
\item[\hskip \labelsep {\bfseries #1}\hskip \labelsep {\bfseries #2.}]}{\end{trivlist}}
\newenvironment{reflection}[2][Reflection]{\begin{trivlist}
\item[\hskip \labelsep {\bfseries #1}\hskip \labelsep {\bfseries #2.}]}{\end{trivlist}}
\newenvironment{proposition}[2][Proposition]{\begin{trivlist}
\item[\hskip \labelsep {\bfseries #1}\hskip \labelsep {\bfseries #2.}]}{\end{trivlist}}
\newenvironment{corollary}[2][Corollary]{\begin{trivlist}
\item[\hskip \labelsep {\bfseries #1}\hskip \labelsep {\bfseries #2.}]}{\end{trivlist}}

\begin{document}

% --------------------------------------------------------------
%                         Start here
% --------------------------------------------------------------

%\renewcommand{\qedsymbol}{\filledbox}

\title{Review Assignment 1}%replace X with the appropriate number
\author{Nalet Meinen \\ %replace with your name
Machine Learning
}

\maketitle

\section{Linear algebra review}

\begin{enumerate}
    \item S = $\{v_1, ... , v_n\}$ be an orthogonal set of non-zero vectors in $R^n$. Prove that the vectors in $S$ are linearly independent.

    \noindent\rule{\linewidth}{1pt}

    We assume a linear combination
    \begin{align*} 
        c_1 v_1 + c_2 v_2 + ... + c_k v_k = 0
    \end{align*}
    We want to show that
    \begin{align*} 
        c_1 = c_2 = ... = 0
    \end{align*}
    The dot product of $v_i$ for each $ i = 1,2, ... , k $:
    \begin{align*} 
        0 &= v_i \cdot 0 \\
          &= v_i \cdot (c_1 v_1 + c_2 v_2 + ... + c_k v_k) \\
          &= c_1 v_i \cdot v_1 + c_2 v_i \cdot v_2 + ... + c_k v_i \cdot v_k
    \end{align*}
    $S$ is an orthogonal set, we have $v_i \cdot v_j = 0 \, \textrm{if} \, i \neq j $, then we have:
    \begin{align*} 
        0 = c_i v_i \cdot v_i = c_i \|v_i\|^2
    \end{align*}
    $v_i$ is nonzero and length $\|v_i\|$ is nonzero, following that $c_i = 0$ \newline
    We conclude that $c_1 v_1 + c_2 v_2 + ... + c_k v_k = 0$ for every $ i = 1,2, ... , k $, so $S$ is \textbf{linearly independent}

    \item Given a square matrix $A \in \mathbb{R}^{n \times n}$ and a vector $x \in \mathbb{R}^n$ show that $x^\intercal Ax = x^\intercal ( \frac{1}{2} A + \frac{1}{2} A^\intercal )x$.
    
    \noindent\rule{\linewidth}{1pt}

    We assume that:

    \begin{align*} 
        A = 
        \begin{bmatrix}
            a_{11}    &   a_{12}    & \dots     &   a_{1n}    \\
            a_{21}    &   a_{22}    & \dots     &   a_{2n}    \\
            \vdots    &  \vdots     & \vdots    &   \vdots    \\   
            a_{m1}    &   a_{m2}    & \dots     &   a_{mn}    \\
        \end{bmatrix}
        x = 
        \begin{pmatrix}
            b_1     &
            b_2     &
            \dots  &
            b_m
        \end{pmatrix}
        \quad \emph{where} \, m = n
    \end{align*}

    The transposed values are:

    \begin{align*} 
        A^\intercal = 
        \begin{bmatrix}
            a_{11}    &   a_{21}    & \dots     &   a_{m1}    \\
            a_{12}    &   a_{22}    & \dots     &   a_{2n}    \\
            \vdots    &  \vdots     & \vdots    &   \vdots    \\   
            a_{1n}    &   a_{2n}    & \dots     &   a_{mn}    \\
        \end{bmatrix}
        x^\intercal = 
        \begin{pmatrix}
            b_1     &
            b_2     &
            \dots  &
            b_m
        \end{pmatrix}
    \end{align*}

    We want to show that this equation is true:
    \begin{align*} 
        x^\intercal Ax = x^\intercal ( \frac{1}{2} A + \frac{1}{2} A^\intercal )x
    \end{align*}

    If we insert the matrices:
    \begin{align*}
        \begin{pmatrix}
            b_1     &
            b_2     &
            \dots  &
            b_m
        \end{pmatrix} 
        \begin{bmatrix}
            a_{11}    &   a_{12}    & \dots     &   a_{1n}    \\
            a_{21}    &   a_{22}    & \dots     &   a_{2n}    \\
            \vdots    &  \vdots     & \vdots    &   \vdots    \\   
            a_{m1}    &   a_{m2}    & \dots     &   a_{mn}    \\
        \end{bmatrix}
        \begin{pmatrix}
            b_1     &
            b_2     &
            \dots  &
            b_m
        \end{pmatrix}
        = 
        \begin{pmatrix}
            b_1     &
            b_2     &
            \dots  &
            b_m
        \end{pmatrix} (\frac{1}{2} 
        \begin{bmatrix}
            a_{11}    &   a_{12}    & \dots     &   a_{1n}    \\
            a_{21}    &   a_{22}    & \dots     &   a_{2n}    \\
            \vdots    &  \vdots     & \vdots    &   \vdots    \\   
            a_{m1}    &   a_{m2}    & \dots     &   a_{mn}    \\
        \end{bmatrix}
         + \frac{1}{2}
        \begin{bmatrix}
            a_{11}    &   a_{21}    & \dots     &   a_{m1}    \\
            a_{12}    &   a_{22}    & \dots     &   a_{2n}    \\
            \vdots    &  \vdots     & \vdots    &   \vdots    \\   
            a_{1n}    &   a_{2n}    & \dots     &   a_{mn}    \\
        \end{bmatrix})
        \begin{pmatrix}
            b_1     &
            b_2     &
            \dots  &
            b_m
        \end{pmatrix}
    \end{align*}
    
\end{enumerate}    

% The implicit representation of a cylinder goes as
% \[ \|x - c\|^2 - r^2 - (x-c) \cdot \vec d = 0 \]

% \noindent Where $\vec d$ is the direction of the cylinder from the center 
% \newline

% \noindent We plug in the ray parameterization from

% \[ x = o + t \cdot d \]

% \noindent therefore we get

% \[ \|o + t \cdot d - c\|^2 - r^2 - (o + t \cdot d - c) \cdot \vec d = 0 \]


% \noindent resulting in the final from

% \begin{align*} t^2 \cdot \|d\|^2 - (\vec d - ( o - c) )^2 \cdot \\
% t \cdot 2 \cdot ((d - (o - c)) - (\vec d - ( o - c) ) \cdot (o - c) ) \cdot \\
% \|(o - c) - (o - c)\|^2 - (\vec d - (o - c))^2 - r^2= 0 
% \end{align*}


\end{document}
